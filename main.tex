\documentclass[12pt, a4paper]{article}

\usepackage[utf8]{inputenc}
\usepackage[german]{babel}			%Sprache auf deutsch setzten
\usepackage{amsmath}
\usepackage{amsfonts}
\usepackage{caption}
\usepackage{graphicx}				%für das Einfügen von Bildern
\usepackage{subfigure}
\usepackage{tabularx, booktabs}
\usepackage{trfsigns}
\usepackage{scrpage2}
\usepackage{ngerman}
\usepackage{german}
\usepackage{setspace}				%Für den Abstand der einzelnen Zeilen
\usepackage{rotating}
\usepackage{comment}
\usepackage{multicol}
\usepackage{wrapfig}
\usepackage{float}

\usepackage[hidelinks]{hyperref}
\usepackage{hyperref}

\usepackage{pgfplots}
\usepackage{nameref}


\usepackage{color}
\definecolor{DarkPurple}{rgb}{0.4,0.1,0.4}
\definecolor{LightLime}{rgb}{0.3,0.5,0.4}
\definecolor{Blue}{rgb}{0.0,0.0,1.0}

\usepackage{listings}
%\usepackage{beramono}						% Text im Codebereich
\lstdefinestyle{matlab}
{
language=Matlab,
columns=flexible,
belowcaptionskip=1\baselineskip,
numbers=left,
xleftmargin=\parindent,
frame=single,
breaklines=true,
frameround=tttt,
showstringspaces=false,						% 
basicstyle=\footnotesize\ttfamily,			% Schriftgröße verkleiner, weichere Darstellung
keywordstyle=\bfseries\color{DarkPurple},	% Farbe für while, for, etc.
commentstyle=\itshape\color{LightLime},		% Farbe für Kommentare
stringstyle=\color{Blue},					% Farbe für Strings
captionpos=b                    			% sets the caption-position to bottom
}

\lstdefinestyle{c}
{
	language=C,
	columns=flexible,
	belowcaptionskip=1\baselineskip,
	numbers=left,
	xleftmargin=\parindent,
	frame=single,
	breaklines=true,
	frameround=tttt,
	showstringspaces=false,						% 
	basicstyle=\footnotesize\ttfamily,			% Schriftgröße verkleiner, weichere Darstellung
	keywordstyle=\bfseries\color{DarkPurple},	% Farbe für while, for, etc.
	commentstyle=\itshape\color{LightLime},		% Farbe für Kommentare
	stringstyle=\color{Blue},					% Farbe für Strings
	captionpos=b                    			% sets the caption-position to bottom
}


\usepackage[top=1.3in, bottom=1.3in, left=1in, right=1in]{geometry}

\begin{document}
		
		\begin{titlepage}
				
				\begin{center}
						
						\includegraphics[width=0.5\textwidth]{HawLogo.png}
						\\[1.5cm]
						\LARGE Digitale Signalverarbeitung
						
						\newcommand{\HRule}{\rule{\linewidth}{0.5 mm}}
						\HRule \\[0.3cm]
						{\huge \bfseries Labor Nr. 3: FIR Filter} \\[0.3cm]
						\HRule \\[1.5cm]
						
						\begin{minipage}{0.4\textwidth}
								\begin{flushleft}
										\large \emph{Autoren:}\\
										Tommy \textsc{Jahnke}\\
										Nils \textsc{Parche}
									\end{flushleft}
							\end{minipage}
						\hfill
						\begin{minipage}{0.4\textwidth}
								\begin{flushright}
										\large \emph{Professor:}\\
										Prof. Dr. \textsc{Vollmer}
										
									\end{flushright}
							\end{minipage}
						
						%\begin{flushleft}
						%	\begin{minipage}{0.4\textwidth}
						%		\large \emph Gruppenmitglieder:\\
						%		J. Sebastian \textsc{Frisch}
						%	\end{minipage}
						%\end{flushleft}
						\vfill
						
						{\large \today}
					\end{center}
				
			\end{titlepage}
		
		\newpage
		\setcounter{page}{1}
		\pagenumbering{roman}
		%\listoffigures
		
		%\newpage
		%\listoftables
		
		\newpage
		\pagenumbering{arabic}
		%\clearpage
		%\thispagestyle{empty}
		%\phantom{a}
		%\vfill
		%\newpage
		%\addtocounter{page}{-2} %Zählt blank page und Inhalt nicht mit in Nummerierung
		
		\pagestyle{scrheadings} %Hier ohne Seitenzahl
		\setheadsepline{0.4pt}	
		\ihead{Jahnke/Parche}
		\ohead{\includegraphics[width=0.215\textwidth]{HawLogo.png}}
		\setfootsepline{0.4pt}
		\ifoot{\today}
		
		\tableofcontents
		\newpage
		\newpage
		
		\setcounter{page}{1}
		
		\pagestyle{scrheadings} %Hier mit Seitenzahl
		\setheadsepline{0.4pt}	
		\ihead{Jahnke/Parche}
		\ohead{\includegraphics[width=0.215\textwidth]{HawLogo.png}}
		\setfootsepline{0.4pt}
		\ifoot{\today}
		\ofoot{\pagemark}
		
		\section{Beschreibung}
Die Labordurchführung wurde nach der Praktikumsbeschreibung \glqq FIR Filter Implementierung in MATLAB und in C\grqq~bearbeitet. In dieser Beschreibung wird davon ausgegangen, dass die Laborbeschreibung vorliegt.\\
In dem nachfolgenden Bericht wird der FIR-Filter theoretisch untersucht und als\\$h_{TP},~h_{HP}~und~h_{BP}$~auf einem DSP Implementiert und untersucht.


		\section{Attachements}
\subsection{Vorbereitung}
In der Versuchsvorbereitung wurden mit Matlab der IIR-Filterentwurf untersucht. Dabei wurden die zwei Filtertypen, elliptisch und Chebychev mit den Spezifikationen siehe Aufgabenbeschreibung simuliert. Für die spätere Implementierung auf einen DSP wurden die resultierenden Übertragungsfunktionen auf Systeme 2. Ordnung reduziert. Die Verstärkungsfaktor g wird dabei auf die einzelnen Systeme 2. Ordnung gleichmäßig aufgeteilt.
\subsubsection{Attachement A}

Die Filtercharakteristik Elliptisch, Chebychev sind in der Allgemeinen und in der Kaskadenstruktur in ihrem Amplitudengang in den nachfolgenden Abbildungen Dargestellt.

\begin{figure}[h]
\centering
\includegraphics[width=0.7\linewidth]{Bilder/Attachment_A_ELLIP}
\caption{IIR-Filter: Elliptischer-Filtertyp - Amplitudengang}
\label{fig:Attachment_A_ELLIP}
\end{figure}
\noindent Amplitudengang des elliptischen Filtercharakteristik. Die maximale und minimale Dämpfung wird sowohl im Passband als auch im Stopband eingehalten.

\clearpage

\begin{figure}[h]
	\centering
	\includegraphics[width=0.7\linewidth]{Bilder/Attachment_A_ELLIP_KASKADE}
	\caption{IIR-Filter: Elliptischer-Filtertyp Kaskadenstruktur - Amplitudengang}
	\label{fig:Attachment_A_ELLIP_KASKADE}
\end{figure}
\noindent Die kaskadierte Form des elliptischen Filter zeigt den gleichen Verlauf auf wie die allgemein Form auf.

\begin{figure}[h]
\centering
\includegraphics[width=0.7\linewidth]{Bilder/Attachment_A_CHEBY}
\caption{IIR-Filter: Chebychev-Filtertyp - Amplitudengang}
\label{fig:Attachment_A_CHEBY}
\end{figure}
\noindent Amplitudenverlauf des Chebychev Filtertyp. Die maximale Sperrdämpfung ist im Vergleich zum elliptischen Filtertyp stärker und steigt nicht mehr im Auflösungsbereich an.

\clearpage

\begin{figure}[h]
\centering
\includegraphics[width=0.7\linewidth]{Bilder/Attachment_A_CHEBY_KASKADE}
\caption{IIR-Filter: Chebychev-Filtertyp Kaskadenstruktur - Amplitudengang}
\label{fig:Attachment_A_CHEBY_KASKADE}
\end{figure}
\noindent Bei der kaskadierten Form des Chebychev-Filtertyp ist der Ripple im Passband stärker als in der allgemeinen Form und ist nicht mehr in den Spezifikationen von 0.1 db.\\

\noindent Im nachfolgenden Listing ist ein Auszug des Matlabskript zur Bestimmung des Filter Koeffizienten und Umwandlung in Systeme 2. Ordnung. Als Hinweis sei hier nochmal erwähnt, dass der Verstärkungsfaktor g bei der Überführung in Systemform 2. Ordnung auf die Koeffizienten der einzelnen Stufen gleichmäßig verteilt wird, um einen Überlauf bei der Implementierung in die Hardware zu verhindern. In der hier eingesetzten Form der Biquad wird die Verstärkung auf die b-Koeffizienten skaliert. Zusätzlich müssen noch alle Koeffizienten auf Eins normiert werden. $b_{k_{normiert}} = b_{k} \cdot 32767$\\

\begin{figure}[h]
\centering
\includegraphics[width=0.7\linewidth]{Bilder/BIQUAD}
\caption{ Realisierung der k-ten Sektion 2er-Ordnung (BIQUAD)}
\label{fig:BIQUAD}
\end{figure}

\clearpage

\lstinputlisting[style=c, caption={IIR-Filter Matlab Skript}, label={lst:fir_2a_koeff}]{Code/IIR_Filter_koeffizienten.m}

\noindent Koeffizienten der Filtertypen: elliptischer und Chebychev\\
Bei den Koeffizienten ist zu erkennen, dass mit einer elliptischen Implementierung zwei Biquads benötigt werden um die Spezifikationen zu erreichen. Mit der Chebychev Umsetzung werden drei Stages benötigt um diese zu erreichen.\\


\lstinputlisting[style=c, caption={IIR-Filter Koeffizienten elliptischer Typ}, label={lst:fir_2a_koeff}]{Code/IIR_ellip_LP.h}

\lstinputlisting[style=c, caption={IIR-Filter Koeffizienten elliptischer Typ}, label={lst:fir_2a_koeff}]{Code/IIR_cheby1_LP.h}

\clearpage

\subsubsection{Attachement B}

\begin{figure}[h]
\centering
\includegraphics[width=0.7\linewidth]{Bilder/Attachment_B_ELLIP}
\caption{}
\label{fig:Attachment_B_ELLIP}
\end{figure}

\begin{figure}[h]
\centering
\includegraphics[width=0.7\linewidth]{Bilder/Attachment_B_ELLIP_KASKADE}
\caption{}
\label{fig:Attachment_B_ELLIP_KASKADE}
\end{figure}

\clearpage

\subsubsection{Attachement D}

\begin{figure}[h]
\centering
\includegraphics[width=0.7\linewidth]{Bilder/Attachment_D_Eingangszeitsignal}
\caption{}
\label{fig:Attachment_D_Eingangszeitsignal}
\end{figure}

\begin{figure}[h]
\centering
\includegraphics[width=0.7\linewidth]{Bilder/Attachment_D_Ausgangssignal_ellip_lp}
\caption{}
\label{fig:Attachment_D_Ausgangssignal_ellip_lp}
\end{figure}

\clearpage

\begin{figure}[h]
\centering
\includegraphics[width=0.7\linewidth]{Bilder/Attachment_D_Ausgangssignal_cheby_lp}
\caption{}
\label{fig:Attachment_D_Ausgangssignal_cheby_lp}
\end{figure}

\begin{figure}[h]
\centering
\includegraphics[width=0.7\linewidth]{Bilder/Attachment_D_Ausgangssignal_ellip_hp}
\caption{}
\label{fig:Attachment_D_Ausgangssignal_ellip_hp}
\end{figure}

\clearpage

\subsubsection{Attachement E}

\begin{figure}[h]
\centering
\includegraphics[width=0.7\linewidth]{Bilder/Attachment_E_Eingangsspektrum}
\caption{}
\label{fig:Attachment_E_Eingangsspektrum}
\end{figure}

\begin{figure}[h]
\centering
\includegraphics[width=0.7\linewidth]{Bilder/Attachment_E_ELLIP_LP_Spektrum}
\caption{}
\label{fig:Attachment_E_ELLIP_LP_Spektrum}
\end{figure}

\begin{figure}[h]
\centering
\includegraphics[width=0.7\linewidth]{Bilder/Attachment_E_CHEBY_LP_Spektrum}
\caption{}
\label{fig:Attachment_E_CHEBY_LP_Spektrum}
\end{figure}

\begin{figure}[h]
\centering
\includegraphics[width=0.7\linewidth]{Bilder/Attachment_E_ELLIP_HP_Spektrum}
\caption{}
\label{fig:Attachment_E_ELLIP_HP_Spektrum}
\end{figure}

\clearpage

\subsection{Lowpass filter}
\subsubsection{Attachement F}

	\begin{figure}[h]
		\centering
		\includegraphics[width=0.8\linewidth]{Bilder/EllipCheby}
		\caption{Blau: Elliptisches IIR TP-Filter | Rot: Chebychev IIR TP-Filter}
		\label{fig:EllipCheby}
	\end{figure}
	
\noindent Der linke Ausgang lieferte das mit einem elliptischen TP-Filter gefilterte Eingangssignal (blau), der rechte Ausgang lieferte das mit einem Chebychev-TP-Filter gefilterte Eingangssignal (rot). Ein Frequenzsweep von 0 .. 4kHz wurde für die Messungen durchgeführt. Die Filterrealisierung geschah jeweils über k kaskadierter Biquads zweiter Ordnung. Für das elliptische Filter war k = 2, für Chebychev war k = 3.

\subsubsection{Attachement G}
\noindent Die Unterschiede in der Implementierung ergeben sich eindeutig:
\begin{itemize}
	\item Elliptisches Filter: 2 Stufen, somit 2 Biquads
	\item Chebychev Filter: 3 Stufen, somit 3 Biquads
\end{itemize}
\noindent Der Rechenaufwand des Chebychev Tiefpasses ist folglich anderthalb mal so groß wie der des Elliptischen Tiefpasses.

\clearpage

\subsection{Hochpass - Tiefpass - Weichenfilter}
\noindent Nun wurde auf dem linken Ausgangskanal das mit einem elliptischen Tiefpass gefilterte Eingangssignal ausgegeben, auf dem rechten Ausgangskanal wurde das mit einem elliptischen Hochpass gefilterte Eingangssignal ausgegeben.

\begin{figure}[h]
	\centering
	\includegraphics[width=0.7\linewidth]{Bilder/EllipTPHP}
	\caption{Blau: elliptischer Tiefpass | Rot: elliptischer Hochpass}
	\label{fig:EllipTPHP}
\end{figure}

\clearpage

\subsection{Modifiziertes elliptisches Tiefpassfilter}
\subsubsection{Attachement H}
\noindent Durch die Verwendung der doppelten Verzögerungszeit $|T| \rightarrow |2T|$ ...

\subsubsection{Attachement J}

\begin{figure}[h]
\centering
\includegraphics[width=0.7\linewidth]{Bilder/Biquad2T}
\caption{Verwendete Biquad-Sektion mit 2 Verzögerungen}
\label{fig:Biquad2T}
\end{figure}

\noindent Es wurde jeweils eine weitere Verzögerung implementiert. Es ergibt sich in C-Syntax:\\
$W_{k0}(n)=W_{k1}(n)$\\
$W_{k1}(n)=b_{k1}*x_k(n))+y_k(n)*(-a_{k1})+W_{k2}(n)$\\
$W_{k2}(n)=W_{k3}(n)$\\
$W_{k3}(n)=b_{k2}*x_k(n)+y_k(n)*(-a_{k2})$

\clearpage

\subsubsection{Attachement K}

	\begin{figure}[h]
		\centering
		\includegraphics[width=0.7\linewidth]{Bilder/ellip2T}
		\caption{Frequenzgang modifiziertes Tiefpassfilter $|T| \rightarrow |2T|$}
		\label{fig:ellip2T}
	\end{figure}
	
\noindent Durch die Verwendung der doppelten Verzögerungszeit $|T| \rightarrow |2T|$ verschiebt sich der Frequenzgang des Tiefpasses. Der gleiche Effekt würde bei der Halbierung der Abtastfrequenz des DSPs auftreten. Der sich periodisch wiederholende Frequenzgang des Filters ist nun dichter zusammengerückt. Wir sehen auf der Abbildung den um $\frac{f_s}{2}$ gespiegelten Frequenzgang. 

		\section{Fazit}

Für die IIR-Filter Koeffizienten Bestimmung und Zerlegung der Übertragungsfunktionen in Systeme 2. Ordnung wurde in dieser Versuchsvorbereitung mehr Aufwand benötigt, als zur Bestimmung der Koeffizienten von FIR-Filtern. Bei dem Filterentwurf ist Matlab eine sehr große Hilfe. Es werden viele Funktionen bereitgestellt mit denen: Filter entworfen und analysiert werden können ohne diese in Hardware einbauen zu müssen. IIR Filter müssen stets auf Stabilität geprüft werden. Durch Konvertierung der Koeffizienten kann der Filter schwingen.\\\\
\noindent Durch die Überführung von Übertragungsfunktionen $H(z)$ in Systeme 2. Ordnung und geschickter Umwandlung in eine Biquad-Form, sind IIR-Filter relativ einfach in einem DSP zu Implementieren. Wichtige Randbedingungen waren unter anderem: Filterkoeffizienten sind zu Normieren, der Allgemeine Verstärkungsfaktor ist auf die führenden Filterkoeffizienten gleichmäßig aufzuteilen und bei Rechenoperationen in den Biquad's muss der Überlauf durch Bit-shifting verhindert werden.\\\\
In der IIR Biquad Filterimplementierung lässt sich ein Filtertype relativ einfach in eine andere Filterart durch einfügen von Verzögerungen überführen. Durch die Ergänzung von Verzögerungen wird das sich periodisch wiederholende Frequenzspektrum dichter zusammengedrückt, sodass aus einem Tiefpass eine Bandsperre wird.\\\\
Im Labor wurden zwei Filterentwurfsmodelle implementiert. Je nach Entwurfsmodell steigt die Koeffizienten Anzahl, was dazuführt das mehr Biquad's benötigt werden. Der elliptische Entwurf, benötigt eine geringere Anzahl an Koeffizienten (2x Biquad) und weißt einen sehr geringen Ripple in Passband auf. Beim chebychev Entwurf werden mehr Koeffizienten benötigt, dadurch ist die Sperrbanddämpfung wesentlich stärker als beim elliptischen Entwurf.\\\\

\begin{table}[h]
	\centering
	\begin{tabular}{c | c}
	FIR & IIR	\\
	\hline
	+ einfacher Filterentwurf	&	+ geringere Anzahl an Koeffizienten \\
	+ Filterimplementierung ist Stabil		&	+ effektive Implementierung mit Biquad's\\
	- viele Koeffizienten	& - nicht zwingend Stabil\\
	\end{tabular}
	\caption{IIR-Filter vs. FIR-Filter}
\end{table}

		\include{Anhang}
	
	%	\bibliography{bibfile}{}
	%	\bibliographystyle{plain}
		
	

\end{document}